% Template: Fabian Wenzelmann, 2016 - 2017

\documentclass[a4paper,
  twoside, % für Unterscheidung gerade / ungerade Seite
  headlines=2.1 % Anzahl der Zeilen im Kopf, wenn man mehr verwenden will erhöhen
  ]{scrartcl}
\usepackage[
  margin=2cm,
  includefoot,
  footskip=35pt,
  includeheadfoot,
  headsep=0.5cm,
]{geometry}
\usepackage[ngerman]{babel}
\usepackage[utf8]{inputenc}
\usepackage{amsmath}
\usepackage{amssymb}
\usepackage[automark,headsepline]{scrlayer-scrpage}
\usepackage{enumerate}
\usepackage[protrusion=true,expansion=true, kerning]{microtype} % Sieht damit einfach schöner aus

\newcommand{\yourname}{Simon Schuler \and Jannik Thoma} % Weitere Autor*innen mittels \and Befehl einfügen
% \newcommand{\yourname}{Dein Name \\ \texttt{123456789} \and Anderer Name \\ \texttt{987654321}} % Hier ein Bsp mit Martikelnummern, auf jeden Fall \headingname anpassen (keine newlines)
\newcommand{\headingname}{\yourname} % Namen wie sie im Kopf erscheinen, wenn zu lang z.B. nur Nachnamen verwenden
% Auch wenn man eine Trennung durch Kommata haben will hier die Namen erneut durch Kommata getrennt wiederholen
% \newcommand{\headingname}{Dein Name, Anderer Name} % Hier ein Beispiel für eine anders formatierte Liste
\newcommand{\lecture}{Technische Informatik WS 2017/18}
\newcommand{\sheetnum}{3}
\newcommand*{\QED}{\hfill\ensuremath{\square}}%
\author{\yourname}
\title{\lecture}
\subtitle{Übungsblatt \sheetnum}
\date{} % Wer ein Datum auf dem Dokument haben will hier eintragen, \today erstellt das heutige Datum

\pagestyle{scrheadings}
\setkomafont{pagehead}{\normalfont}
\lohead{\lecture\\\headingname}
\lehead{\lecture\\\headingname}
\rohead{Übungsblatt \sheetnum}
\rehead{Übungsblatt \sheetnum}


\begin{document}

\maketitle

% Hier kommt dein eigentlicher LaTeX Code
\section*{Aufgabe 1}
    
\section*{Aufgabe 2}
  
\section*{Aufgabe 3}
  
  Sei \(a \in \mathds{B}^{n}, a = a_{n-1}, ...., a_{0} \Rightarrow [a]=[a_{n-1}a]\)\\
  Zweierkomplement sei definiert als: 
  \begin{flalign*}
  [a] = -a_{n-1} \cdot 2^{n-1} + \sum_{i=0}^{n-1}a_i2^i
  \end{flalign*}
  Beweis der oberen Annahme:
  \begin{flalign*}
  [a_{n-1}a] &= (-a_{n-1} \cdot 2^n) + \sum_{i=0}^{n-1}a_i2^i \\
  &= -a_{n-1} \cdot 2^n + (a_{n-1}\cdot 2^{n-1}+ \sum_{i=0}^{n-2}a_i2^i) \\
  &= (-2\cdot a_{n-1}+a_{n-1})\cdot 2^{n-1}+\sum_{i=0}^{n-2}a_i2^i\\
  &= a-a_{n-1}\cdot 2^{n-1}+\sum_{i=0}^{n-2}a_i2^i \\
  &=[a]
  \end{flalign*}

\section*{Aufgabe 4}

\section*{Aufgabe 5}
		(a)\begin{flalign*}
			(\sum_{i=-k}^{n-1}2^i \cdot 0)-0 \cdot (2^n-2^{-k}) &= 0 \\
			(\sum_{i=-k}^{n-1}2^i \cdot 1) -1 \cdot (2^n - 2^{-k} &= 0
		\end{flalign*}
		Die Symmetrie des Einerkomplements zeigt, dass es zwei Darstellungen der '0' gibt: Zum einen alle Bits auf '0' gesetzt, andererseits alle auf '1'. \\
		(b) Addiert man $d$ und das Komplement $d'$ erhält man die größte Zahl, also '0' im Einerkomplement. \\
		Z.z.: $[d]+[d'] = 0$ \\\\
		Fall 1: \\
		\begin{align*}
			&(\sum_{i=-k}^{n-1}2^i \cdot 1)-0\cdot (2^n-2^{-k})+(\sum_{i=-k}^{n-1}2^i \cdot 0) -1 \cdot (2^n - 2^{-k}) =0 \\
			&= (\sum_{i=-k}^{n-1}2^i \cdot 1) -1 \cdot (2^n - 2^{-k}) = 0
		\end{align*} \\
		Fall 2:\\
		\begin{align*}
			&(\sum_{i=-k}^{n-1}2^i \cdot 0)-1 \cdot (2^n-2^{-k}+(\sum_{i=-k}^{n-1}2^i \cdot 1)-0\cdot(2^n-2^{-k}) = 0 \\ 
			&=-1\cdot(2^n-2^{-k}) +(\sum_{i=-k}^{n-1}2^i \cdot 1) = 0
		\end{align*}
		
\end{document}
