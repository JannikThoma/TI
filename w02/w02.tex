% Template: Fabian Wenzelmann, 2016 - 2017

\documentclass[a4paper,
twoside, % für Unterscheidung gerade / ungerade Seite
headlines=2.1 % Anzahl der Zeilen im Kopf, wenn man mehr verwenden will erhöhen
]{scrartcl}
\usepackage[
margin=2cm,
includefoot,
footskip=35pt,
includeheadfoot,
headsep=0.5cm,
]{geometry}
\usepackage[ngerman]{babel}
\usepackage[utf8]{inputenc}
\usepackage{amsmath}
\usepackage{amssymb}
\usepackage[automark,headsepline]{scrlayer-scrpage}
\usepackage{enumerate}
\usepackage[protrusion=true,expansion=true, kerning]{microtype} % Sieht damit einfach schöner aus

\newcommand{\yourname}{Simon Schuler \and Jannik Thoma} % Weitere Autor*innen mittels \and Befehl einfügen
% \newcommand{\yourname}{Dein Name \\ \texttt{123456789} \and Anderer Name \\ \texttt{987654321}} % Hier ein Bsp mit Martikelnummern, auf jeden Fall \headingname anpassen (keine newlines)
\newcommand{\headingname}{\yourname} % Namen wie sie im Kopf erscheinen, wenn zu lang z.B. nur Nachnamen verwenden
% Auch wenn man eine Trennung durch Kommata haben will hier die Namen erneut durch Kommata getrennt wiederholen
% \newcommand{\headingname}{Dein Name, Anderer Name} % Hier ein Beispiel für eine anders formatierte Liste
\newcommand{\lecture}{Technische Informatik WS 2017/18}
\newcommand{\sheetnum}{2}
\newcommand*{\QED}{\hfill\ensuremath{\square}}%
\author{\yourname}
\title{\lecture}
\subtitle{Übungsblatt \sheetnum}
\date{} % Wer ein Datum auf dem Dokument haben will hier eintragen, \today erstellt das heutige Datum

\pagestyle{scrheadings}
\setkomafont{pagehead}{\normalfont}
\lohead{\lecture\\\headingname}
\lehead{\lecture\\\headingname}
\rohead{Übungsblatt \sheetnum}
\rehead{Übungsblatt \sheetnum}


\begin{document}
	
	\maketitle
	
	% Hier kommt dein eigentlicher LaTeX Code
	\section*{Aufgabe 1}
	
	\section*{Aufgabe 2}
	
	\begin{tabular}{l l|l}
		0 & LOADI 4	& ACC = n = 4 Testwert, exp. fib(4) = 5 \\
		1 & STORE 30& S(30) = ACC = n = 4 \\
		
		2 & LOADI 1	& ACC = 1\\
		3 & STORE 33 & Initialisiere zwei "Zwischenspeicher" S(32)/S(33)\\
		4 & STORE 32 & für n-1 und n-2\\
		5 & LOAD 30	& Lade S(30) = n in ACC\\
		6 & JUMP EQ 9 & Wenn S(30) = n = 0 beende das Programm\\
		7 & SUBI 1 & Dekrementiere (n-1) ACC = n\\
		8 & STORE 30 & S(30) = ACC = n\\
		9 & LOAD 33	& ACC = S(33)\\
		10 & ADD 32 & Addiere beide Zwischenspeicher\\
		11 & STORE 33 & S(33) += S(32)\\
		12 & SUB 32 & Summe - (n-2)\\
		13 & STORE 32 & Schreibe (n-1) in (n-2) und n in (n-1)\\
		14 & JUMP -9 & Weiter ab Schritt 5\\
		15 & END &
	\end{tabular}
	
	\section*{Aufgabe 3}
	
	\section*{Aufgabe 4}
	
	\section*{Aufgabe 5}
	
\end{document}
